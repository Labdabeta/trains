\documentclass{article}

\usepackage[indentafter]{titlesec}
\titleformat{name=\section}{}{\thetitle.}{0.8em}{\centering\scshape}
\titleformat{name=\subsection}[runin]{}{\thetitle.}{0.5em}{\bfseries}[.]
\titleformat{name=\subsubsection}[runin]{}{\thetitle.}{0.5em}{\itshape}[.]
\titleformat{name=\paragraph,numberless}[runin]{}{}{0em}{}[.]
\titlespacing{\paragraph}{0em}{0em}{0.5em}
\titleformat{name=\subparagraph,numberless}[runin]{}{}{0em}{}[.]
\titlespacing{\subparagraph}{0em}{0em}{0.5em}

\usepackage{tikz}
\usepackage{graphicx}
\usepackage{hyperref}

\usepackage[margin=1in]{geometry}

\title{CS 452 Kernel 2}
\author{Louis A. Burke (laburke) and Taras Kolomatski (tkolomat)}
\date{\today}

\begin{document}

\begingroup
\let\newpage\relax%
\maketitle
\endgroup

\section*{Overview}

Early childhood development is markedly impacted by the name given to a child by
its parents. This moniker is imbued on the child not by any material process,
but by repeated, direct, and structured communication. While a child may make
noise to communicate to anyone who will listen, the adults utilize these sigils
to directly and efficiently pass ideas to each other. By means of repeated
messages, parents emboss a child's name on it from an early age. When the child
has grown they have internalized their handle to such a degree that it becomes
tied to their identity. Humanity is thus naming and messaging.

Our realtime kernel now also supports naming, making use of an underlying
messaging system to set the names of tasks and associate them with their task
IDs.

\section*{Structure}

\subsection*{Message Passing}

The message passing functionality is merely added to the existing scheduler
functionality. Negative priorities are now used to indicate blocked tasks, so
that the scheduler knowns not to reschedule them when it ignores them on the top
of the active queue. In order to know if a blocked task has to be rescheduled or
not, the scheduler marks each task with whether it is currently in the
scheduling system or not.

As for message passing itself, each task descriptor now contains a queue of
tasks waiting to send to it. When it calls receive it will accept one of those
tasks and advance the queue. If the queue is empty it will block until it is
not.

As for sending, tasks merely contain two pointers to buffers that can be written
to by other tasks. Replying is pretty similar to send, except it also has to
unblock the target.

The way we setup our kernel interface, kernel commands can only take 3
arguments. This allows all arguments to be passed in registers, but is not
enough arguments for the message passing calls. To solve this we had two
options. The first was to arbitrarily wrap all arguments in a struct and pass a
pointer to it, but that seemed clunky. Instead we created a buffer utility type.
By replacing buffer/size pairs with a single buffer pointer we reduce even send
to just three arguments. This also allows us to write the optimized and advanced
(truncation checks) buffer copier in a separate module, with minimal overhead.

\subsection*{Name Server}

Our nameserver is just a user task like any other. It is created early and takes
up only a very small task space, but runs at high priority. The name server is
fairly rudimentary, as it doesn't need to deal with a large dynamic load.

It consists of a buffer of stored names and their associated task ids. This
buffer is filled from the top down with names as they are reported to the task.
Instead of searching for old names to replace in the buffer, newly registered
tasks merely continue to write names lower in the buffer. Meanwhile when
searching for a name, the buffer is traversed from low to high, in reverse
order. In this manner the newer names will always register before the older
ones.

While this leaves us with a fixed total number of "RegisterAs" calls, the number
we are allowed is easily modified at compile-time. Since we can know in advance
exactly how many named services we will need, we can precompute this value
reliably.

\section*{Game Task Priorities}

The server was given a priority of 1, while the clients were given priorities of
4 so as to allow the server time to register itself and setup its state. Servers
can in general be a higher priority task than the tasks they serve since they
spend most of their time blocked. Their runtime is simply an extension of that
of the tasks they serve.

\section*{Results}

\begin{center}
\begin{tabular}{c|c|c|c|c}
Length & Caches & SRR & O2 & Time \\ \hline
4 & off & yes & off & 265 \\
64 & off & yes & off & 401 \\
4 & on & yes & off & 19.8 \\
64 & on & yes & off & 29.0 \\
4 & off & no & off & 356 \\
64 & off & no & off & 478 \\
4 & on & no & off & 25.8 \\
64 & on & no & off & 35.0 \\
4 & off & yes & on & 101 \\
64 & off & yes & on & 139 \\
4 & on & yes & on & 8.07 \\
64 & on & yes & on & 10.5 \\
4 & off & no & on & 143 \\
64 & off & no & on & 176 \\
4 & on & no & on & 10.7 \\
64 & on & no & on & 13.2
\end{tabular}
\end{center}

Based on the reward for optimizing different sections of the code, we believe
that the main performance bottlenecks likely reside in three locations.

The first, and hardest to mitigate, is in data copying. When optimization is
turned on, we over-copy. We assume everything is aligned to 4-byte boundaries
and copy as such. This leads to some bugs when optimizations are enabled, but
provides a significant speed up by avoiding an unnecessary branch and divide
(checking the modulo of the size).

Second is the scheduler. When a real-time task sits active it is very fast,
however whenever any task with a lower priority needs to run, we have to shift
the scheduler stack. This shift was very costly when there were a full 32
priorities, so we reduced them to just 8. The effect was pretty small, but still
noticeable.

Finally mode switching itself is still a bottleneck.
Despite involving mostly hand-written assembly, these context switches happen so
frequently and incur such frequent cache misses, that they are a large amount of
time sink. In order to improve this, we optimized a few key instructions. We
also inspected the elf file to see that the assembly instructions were being
placed far away from the rest of the instructions. This lead to a full
organization of the linker script in order to place critical functions near each
other, and long functions near the end.

\section*{Game Task Output}

The output of the game is:

\begin{verbatim}
src/tasks/rps_client.c:27        Joined
src/tasks/rps_client.c:27        Joined
Signed up!
Signed up!
P1: R
P2: S
p2: I lost.
P2: P
P1: I won!
Press any key to continue...
P1: P
p1: Great minds think alike.
Press any key to continue...
P1: S
p2: Great minds think alike.
P2: P
p2: I lost.
P2: Q
p1: I won!
Press any key to continue...
P1: R
src/tasks/rpsserver.c:94        QUIT
p1: Sore loser.
Press any key to continue...
\end{verbatim}

Note that first the clients join and sign up using the name server and the game
server (name server debug shown as expanded debug message). Next they play a few
rounds, announcing their moves (capital letters) and sending them to the server.
When they get a response they parse it and emote as appropriate. They don't get
rescheduled again until they send their next move to the server, so their moves
come immediately after they find out what happened. The three cases of winning,
losing and tying are shown. Finally one of the tasks quits. The other task
continues playing but the server notifies it that the other player has quit
(again shown using expanded debug messages). The result is that the player also
quits, ending the program.

\section*{SHA of commit}

The commit hash of the submission commit (which is on master) is:

\texttt{{{{commit hash}}}}

The repository can be cloned from:

\url{gitlab@git.uwaterloo.ca:laburke/cs452_kernel.git} (SSH)

or

\url{https://git.uwaterloo.ca/laburke/cs452_kernel.git} (HTTPS)

\end{document}
