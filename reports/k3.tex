\documentclass{article}

\usepackage{titlesec}
\titleformat{name=\section}{}{\thetitle.}{0.8em}{\centering\scshape}
\titleformat{name=\subsection}[display]{}{\thetitle.}{0.5em}{\bfseries}
\titleformat{name=\subsubsection}[runin]{}{\thetitle.}{0.5em}{\itshape}[.]
\titleformat{name=\paragraph,numberless}[runin]{}{}{0em}{}[.]
\titlespacing{\paragraph}{0em}{0em}{0.5em}
\titleformat{name=\subparagraph,numberless}[runin]{}{}{0em}{}[.]
\titlespacing{\subparagraph}{0em}{0em}{0.5em}

\usepackage{tikz}
\usepackage{graphicx}
\usepackage{hyperref}

\usepackage[margin=0.7in]{geometry}

\title{\vspace{-3em}CS 452 Kernel 3}
\author{Louis A. Burke (laburke) and Taras Kolomatski (tkolomat)}
\date{\vspace{-0.9em}\today}

\begin{document}

\begingroup
\let\newpage\relax%
\maketitle
\endgroup

\section*{Overview}

\textsc{The} beginning of the industrial revolution was marked by the development of
steam power, enabling manufacturing and locomotion. Whereas an old Dutch sailor
was compelled into an aesthetic contemplation he neither understood nor desired,
face to face for the last time in history with something commensurate to his
capacity for wonder, the inertial march of progress proved psychologically
overwhelming. Post the revelations of Darwin, Freud, and Einstein, how could one
maintain a static view of the context of humanity, the control one could exert
on one's environment, or the continuity of mind and time. Literature reflected
this, deforming mind and time in postmodern work, from the dreamscape of
\textit{Finnegans Wake} to the dissociation of Slothrop in \textit{Gravity's
Rainbow}. Indeed, one could not walk to a room in which the women come and go,
speaking of Michelangelo, without the worries of eating a peach and disturbing
the universe interrupting one's narrative.

In our third kernel milestone we forsake static deterministic execution, opening
ourselves the overwhelming interruptions of the exterior world. Additionally our
kernel now allows us to time things, delaying critical execution paths until the
time is right for them to reactivate. To deal with the large amount of blocking
that is now present, an idle task has also been created.

\section*{Structure}

\subsection*{Notifiers}

\textsc{One} of the major changes to the code is the delegation of some tasks as
notifiers. These tasks spend most of their time event blocked and are directly
linked to the hardware interrupt handler which reactivates them. This introduces
a level of separation between hardware interrupts and the tasks that depend on
them.

A new system call was created in order to facilitate these notifiers and still
allow the kernel to determine when it is safe to return.

Alongside this system call, a new set of system calls similar to those of
\texttt{Send}/\texttt{Receive}/\texttt{Reply} were created. These new system
calls \texttt{Share}/\texttt{Obtain}/\texttt{Respond} merely share a raw pointer
with another task rather than copying data. This is very beneficial in a few
cases where read-only data is passed between tasks. While this still incurs a
cache miss, it does not have to copy the memory which significantly improves
performance.

\subsection*{Hardware Interrupts}

\textsc{The} major design decision that we made regarding handling hardware
interrupts was to not pass through the kernel, thus not allowing for
rescheduling. Our HWI handler is a four line C function, exploiting all the
magic of \texttt{gcc} attributes. This function takes two lines to disable the
interrupt (the constant is folded in the compiled assembly), and two more to
check that the notifier is event blocked, and if so, to call an inlined function
that unblocks the notifier. We used the
\texttt{\_\_attribute\_\_((interrupt("IRQ")))}. As our internal function call is
inlined, \texttt{gcc} will know exactly which registers were clobbered in the
HWI handler, and will thus save only those registers to the stack while being
careful to avoid erasing the scratch and argument registers. This attribute
further adds a~ \texttt{\^} to the final \texttt{ldfm} command as to set the
\texttt{cpsr} when restoring the \texttt{pc}. The assembly for the HWI handler
is 31 lines in \texttt{O2}, but that is longer than any execution path through
it (although it's the potentiality for cache misses that makes or breaks
performance). This handful of instructions is all that happens between when a
HWI is raised while in user mode, and the return to the user task.

\subsection*{Idle Task and Timing}

\textsc{The} idle task is not a real task like other user tasks. It is never
present in the scheduler and will never be scheduled by normal execution.
However, when the scheduler returns that it cannot find any active tasks the
kernel explicitly activates a hard-coded idle task. This idle task does nothing
but call the system call \texttt{Pass} to return control to the kernel.

In order to time execution, the kernel activates the 40-bit debugging clock as
soon as it starts. It then calculates how long the kernel took to initialize and
begins timing a number of different values.

When the scheduler returns a task, the kernel saves the elapsed time as kernel
time, then activates that task. When that task interrupts back to the kernel, it
records the elapsed time as user time as well as recording it in the task. Then
the kernel handles the interrupt and records the time it takes as handler time.
This repeats until the scheduler does not return a task.

When the scheduler has no tasks the elapsed kernel time is instead recorded as
idle time. In this way the repeated failed scheduling attempts make up the bulk
of the idle work of the system alongside the recording of the timer values.

When full optimization options are enabled (with the ``make prod'' compilation)
this kernel benchmarking system is not enabled. The reason these recordings are
disabled is because they are fairly costly, especially since they involve
operations on 64-bit integers. In order to benchmark the kernel even with
optimizations enabled there is another compilation option (``make prodebug'')
which compiles with optimization but still enables the ``\texttt{DEBUG\_MODE}''
flag.

\subsection*{Scheduling}

\textsc{Another} optimization performed in this version of the kernel was an
update to the scheduler we use. One of the problems with the previous algorithm
was that its runtime grew quadratically with the highest priority (lowest
priority number) task. Since the idle task effectively has a priority of one
higher than the maximum priority, it would cause the scheduler to spend a lot of
time just rescheduling the idle task.

The main cause of this was the time taken to ``shift'' each scheduling queue up
one level. This required a loop over each queue. To prevent this, the scheduler
structure was expanded from a small list of pointers to a large array of
pointers and queues. While it now takes $O(n^2)$ space and $O(n^2)$ time on
initialization, it only takes $O(1)$ time to reschedule a task (where $n$ is the
number of priorities, it has always been $O(1)$ in number of tasks and thus
$O(1)$ during runtime).

Additionally we found that the compiler refused to inline the rescheduling
function. Unfortunately that function is one of the most called functions in the
entire code-base and the extranneous stack manipulation is a significant
hindrance to performance. Upon closer inspection it was discovered that the
reason the compiler refused to inline the function was because it was written
recursively.

By manually unrolling the recursion into a loop we were finally able to get the
compiler to fully inline the rescheduling process. This significantly improved
the performance of our scheduler even beyond the performance it had before.
Additionally this actually reduced the size of the code as it turns out to have
marginally simplified the generated assembly.

\section*{Results}

\textsc{The} output of the tasks is rather long, consisting of over 50 lines. As
such we will not be printing it in its entirety here since we would be likely to
make a mistake during the copying process. Instead the program can be run and
the output inspected manually.

The order of the output is predominantly determined by the delays of the tasks.
Since the priority 3 task's delay is much shorter, it prints more often. As for
when two tasks would unblock at the same time it would depend on their
priorities which executed first. This could be tested by removing a single tick
from the lowest priority task.

Finally the code prints the number of ticks on the 40-bit timer that elapse in
various parts of execution. You may notice that the total time is less than the
sum of all of the time parts, the reason for this is because it does not account
for the time it takes the kernel to initialize. This is additional useful
debugging information, but does not provide insight into the ``steady-state'' of
the kernel. As such only time since the first task is scheduled is counted
towards total time.

Finally at the end we divide the time in the idle task by the total time to come
up with a percentage of time spent idle. While this is not fully deterministic,
we have found that with optimizations turned on we reliably get a 96\% idle
time.

\section*{SHA of commit}

\textsc{The} commit hash of the submission commit (which is on master) is:

\texttt{{{{commit hash}}}}

\noindent The repository can be cloned from:

\url{gitlab@git.uwaterloo.ca:laburke/cs452_kernel.git} (SSH)

or

\url{https://git.uwaterloo.ca/laburke/cs452_kernel.git} (HTTPS)

\textbf{There is a README in the root directory of the repository which outlines
the loading command.} To run the program once it is compiled just restart the
machine and run \texttt{load -b 0x100000 -h 10.15.167.5
"ARM/path/to/kernel.elf"} to load it. Note that the loaded address is different
from the default value.
\end{document}
