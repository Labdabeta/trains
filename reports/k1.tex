\documentclass{article}

\usepackage[indentafter]{titlesec}
\titleformat{name=\section}{}{\thetitle.}{0.8em}{\centering\scshape}
\titleformat{name=\subsection}[runin]{}{\thetitle.}{0.5em}{\bfseries}[.]
\titleformat{name=\subsubsection}[runin]{}{\thetitle.}{0.5em}{\itshape}[.]
\titleformat{name=\paragraph,numberless}[runin]{}{}{0em}{}[.]
\titlespacing{\paragraph}{0em}{0em}{0.5em}
\titleformat{name=\subparagraph,numberless}[runin]{}{}{0em}{}[.]
\titlespacing{\subparagraph}{0em}{0em}{0.5em}

\usepackage{tikz}
\usepackage{graphicx}
\usepackage{hyperref}

\usepackage[margin=1in]{geometry}

\title{CS 452 Kernel 1}
\author{Louis A. Burke (laburke) and Taras Kolomatski (tkolomat)}
\date{\today}

\begin{document}

\begingroup
\let\newpage\relax%
\maketitle
\endgroup

\section*{Overview}

The higher cognitive functions of the human mind are conjectured to be enabled
due to the structure that language provides. Society hence endows infants with
both the ability to form organized thoughts and to communicate them.
Conversely, the discourse of individuals constitutes a society; a strange loop
is formed, enforcing the sporadic breakthroughs of eons past via positive
feedback. Society is thus structure in the present, which encodes its history in
structural invariants that provide pushback forces allowing a low entropy state
to be maintained. Society is thus context and control.

Having established communication in assignment zero, we move on the the problems
of context and control in our first kernel milestone. A program is a sequence of
instructions that modify state. Many instructions are only meaningful in the
context and assumptions of its past instructions. A single misplaced command or
erased register value is all that is required to derail execution. A kernel
virtualizes the CPU, allowing for the peaceful, indeed synergistic coexistence
of multiple tasks.

Our realtime kernel dynamically manages tasks on a TS-7200 board with a Cirrus
system-on-chip and an ARM core. It provides a scheduler and a system call
interface to allow tasks to be easily written. Being a real-time kernel it must
allow tasks with sufficiently high priority to be given the execution time they
need to do what they have to.

\section*{Structure}

\subsection*{Architecture}

Our kernel is currently a micro kernel, any drivers or other modules will have
to be implemented as separate tasks. It currently consists almost entirely of a
single core loop of execution. This loop transfers control to an active task
using a context switch. It also incorporates a scheduler to determine which task
should be executed. Finally a software interrupt handler is installed in order
to provide kernel functions to user tasks.

The tasks themselves are loaded into hard-coded memory locations. There are 5
sizes of tasks. There are 4 "giant" tasks which are provided with 2mb of stack
each.  There are 14 "big" tasks which are provided with 1mb of stack each. There
are 24 "normal" tasks which are provided with 256kb of stack each. There are 28
"small" tasks which are provided with 64kb of stack each. Finally there are 16
"tiny" tasks which are provided with 16kb of stack each. This takes a full 30
megabytes of space, but the default load address of 0x218000 eats away at over 2
of the 32 megabytes. In order to combat this, we instead load the kernel at
0x100000. We then allow all addresses up to 2mb to be used for kernel code. The
tasks are given task space above this. To avoid stomping on RedBoot's stack we
set the kernel's stack to the first giant task space and reset its stack address
to that of some variable already known to be within the kernel's stack. This
does mean that if the kernel were ever to enter the scheduler it would have the
wrong stack, but since the kernel never leaves the "zombie" priority of -1, it
will never happen.

\subsection*{Context Switch}

To transfer control to and from the kernel, we execute handwritten assembly
constituting the context switch. The state of a user task is stored between the
user task's stack and its \texttt{TaskDescriptor} in kernel space. The
TaskDescriptor contains the saved \texttt{sp}, the contents of the \texttt{spsr}
at the last trap out of the task's user space, and the return value of the last
system call made by the task. The top of the user's stack contain the saved
values of \texttt{r4-r12, lr} followed by the saved \texttt{pc}. To trap out of
the kernel we do the following:
\begin{enumerate}
    \item \textit{Pass the \texttt{sp}, \texttt{spsr}, and \texttt{rval} into
        the \texttt{asm\_EnterTask} function (this places the arguments in
        registers),}
    \item \textit{Restore the task's \texttt{spsr}}
    \item \textit{Store \texttt{r4-r12, lr} on the kernel's stack,}
    \item \textit{Pop the saved \texttt{pc} from the user's stack to the
        $\texttt{lr}_\texttt{svs}$ register (this increments \texttt{r0} where
        the argument is stored),}
    \item \textit{Mask \texttt{cpsr} to switch to system mode}
    \item \textit{Restore the user \texttt{sp}}
    \item \textit{Restore the user registers \texttt{r4-r12, lr}}
    \item \textit{Mask \texttt{cpsr} to switch to supervisor mode}
    \item \textit{Pass the return value into \texttt{r0}}
    \item \textit{\texttt{movs pc, lr}}
\end{enumerate}

Notice that beginning at step 4, we have definitively switched from working in
kernel memory to working in user memory. If we were working in a one line
cache, trapping out of the kernel would induce at most one cache miss. This is
made possible by storing everything that we need from the
\texttt{TaskDescriptor} in registers via argument passing.

When in user space, the active task yields control to the kernel by making a
system call. System calls have a number, the code, that identifies the call
type, and zero or more arguments. The only syscall with arguments in K1 is
\texttt{Create}. The system call functions are wrappers for a mixed C/arm
function that pushes the code and arguments on the user stack and issue an
\texttt{swi}. At kernel startup, we write the address of a assembly function,
\texttt{asm\_EnterKernel}, to the trap table entry associated with software
interrupts. This function performs the following operations:

\begin{enumerate}
    \item \textit{Copy into \texttt{r3}, the value of the current link register,
        which is the saved stack pointer at the time of the \texttt{swi},}
    \item \textit{Mask \texttt{cpsr} to switch to system mode (this gives us
        access to $\texttt{sp}_\texttt{sys}$)}
    \item \textit{Copy \texttt{sp}$_\texttt{sys}$ into \texttt{r1} as to save
        it's value before by the subsequent pushes,}
    \item \textit{Store $\texttt{r4-r12, lr}_\texttt{sys}$ on the user space
        stack,}
    \item \textit{Store \texttt{r3}, and hence the saved \texttt{pc}, on the
        user's stack,}
    \item \textit{At this point, \texttt{r4-r7} are scratch registers; we use
        them to store the syscall arguments from the user stack, starting at the
        address currently in \texttt{r1},}
    \item \textit{Copy the saved \texttt{sp} into \texttt{r0} to return,}
    \item \textit{Mask \texttt{cpsr} to switch to supervisor mode,}
    \item \textit{Save the syscall arguments in a kernel data structure (the
        address is written in code space at startup),}
    \item \textit{Copy the \texttt{spsr} into \texttt{r1} to return it}
    \item \textit{Restore \texttt{r4-r12, lr} from the kernel's stack, but place
        \texttt{lr} in the \texttt{pc} to return from  \texttt{asm\_EnterTask}.}
\end{enumerate}

We switch from working in user memory to kernel memory at step 10. Once again,
there is a single dividing point. Upon returning to the function that called
\texttt{asm\_EnterTask}, we will use the values in \texttt{r0,r1} to save the
\texttt{sp} and saved program status register in the \texttt{TaskDescriptor}.
These two functions exemplify the $ABB^{-1}A^{-1}$ design pattern.

\subsubsection*{The Ersatz Trap Frame}

The context switch allows us to reenter a task after having previously exited it
via a software interrupt. Indeed, there must be a user state that is to be
restored on the user space stack. Yet, we are able to reuse this infrastructure
to enter a newly created task. We do this by writing a \textit{ersatz trap
frame} to the user stack during creation. From the kernel's perspective, we
enter a new task \textit{in medias ras} as though returning from an interrupt.

\subsection*{Scheduler}

Our scheduler is a modified implementation of the O(1) scheduler found in older
versions of the linux kernel.

This scheduler currently consists of one stack for each provided priority
(0-31). Stacks are used instead of queues to simplify execution and reduce
memory footprint. These stacks do \textbf{not} correspond to the distinct
priority levels directly. A task of any priority may eventually appear in any of
the stacks.

When asked to schedule the next task, the scheduler merely returns the top of
the 0th stack. Meanwhile it moves the task it returns to the top of the stack
corresponding to it's priority. Thus if it has priority 10, it will be moved to
the 10th stack. Importantly this means that if a task has priority 0 it will be
placed directly onto the active stack to be popped immediately thereafter. As
such priority 0 is reserved for tasks that must operate in real-time. These
tasks will be scheduled immediately and no other tasks will be scheduled until
they complete. Additionally this rescheduling happens extremely quickly,
requiring only a handful of instructions. The effective C-code for a real-time
task's rescheduling is:

\begin{verbatim}
int irrelevant = 1;
while (irrelevant) {
    struct TaskDescriptor *ret = state->exhausted[0];
    if (ret) {
        state->exhausted[0] = ret->next;

        if (ret->priority > 0) {
            ret->next = *state->exhausted;
            *state->exhausted = ret;
            return ret;
        }
    }
}
\end{verbatim}

Meanwhile if the 0th stack does not have an active task more work is
done. It is here that this method performs slower than a round robin, but since
by definition there must not be any real-time priority tasks left, taking a few
extra cycles is not a problem. At this point each stack is moved one index
lower, while the current 0th "active" stack is moved to the last stack. This
cycles the tasks up one level of priority. In this manner a "highest" priority
task (not real-time, but highest "fair" priority) will be rescheduled to the
very next stack to be loaded. However a lowest priority task will have to wait
for the 0th stack to empty a lot of times in order to get a single chance to
run, after which it must wait again.

Of course this makes the scheduler horrendously inefficient in the case that a
single low priority task is the only one running. However, since low priority
tasks by definition don't need the scheduler to run quickly, this is not a
problem.

\subsection*{Main Loop and System Calls}

The main loop starts by initializing all of the task objects on its giant stack.
It then sets up the first user task with priority 1 and adds it to the
scheduler. Finally, the main loop is run. As long as the scheduler has a task
ready to run we enter said task and wait for it to call a software interrupt.
When it does we analyze the stack to determine what system call was sent and
what its arguments were. Then we enact whatever is requested.

\section*{Results}

The output of our code is:
\begin{verbatim}
Created: 4
Created: 5
TID: 6  PID: 1
TID: 6  PID: 1
Created: 6
TID: 6  PID: 1
TID: 6  PID: 1
Created: 6
FirstUserTask: exiting
TID: 4  PID: 1
TID: 5  PID: 1
TID: 4  PID: 1
TID: 5  PID: 1
\end{verbatim}

The explanation is as follows:

    The user task (id 1) calls create on the two low priority tasks (here 4 and
5).  Since these tasks are lower priority than it, they don't get rescheduled
and the initial task runs again.  Now it calls create to create a high priority
task (here 6).  Since this task is a higher priority than it, it immediately
executes, and since the priority we set it was real-time, it runs to completion
before the first task runs again. As such the entire task runs printing its id
(6) and its parent id (1) to the screen. Now control returns to the first task
and it calls create again. Since task 6 already completed, that id is available
for use again, so the exact same id is re-used and the real-time task runs to
completion identically. Now control returns to the first user task, which exits.
This leaves only the low priority tasks to run. As they are the same priority,
and it is low, they take turns making one system call at a time. Therefore their
outputs are interweaved.

\section*{SHA of commit}

The commit hash of the submission commit (which is on master) is:

\texttt{{{{commit hash}}}}

The repository can be cloned from:

\url{gitlab@git.uwaterloo.ca:laburke/cs452_kernel.git} (SSH)

or

\url{https://git.uwaterloo.ca/laburke/cs452_kernel.git} (HTTPS)

\end{document}
